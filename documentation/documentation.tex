%%%%%%%%%%%%%%%%%%%%%%%
% Author: Ibrahima Barry
%%%%%%%%%%%%%%%%%%%%%%%

% This portion of the LaTeX document are configuration 
% You can see it as all the equivalent of "#include" in C++
\documentclass[12pt]{article}

% These three lines add some packages to the LaTeX.
% This is the equivalent of #include from C++ (importing known libraries)
% The libraries help inserting figures, use math statements, and so on
\usepackage{epsfig}
\usepackage{amsmath,amsthm}
\usepackage{amsfonts}
\usepackage{listings}
\usepackage{url}
\usepackage{graphics}
\usepackage{pgfplots, pgfplotstable}
\pgfplotsset{width=10cm,compat=1.9}
\usetikzlibrary{calc}
\graphicspath{ {./images/} }
\usepackage{geometry}
\geometry{
    a4paper,
    total={170mm, 257mm},
    left=12mm,
    right=12mm,
    top=20mm,
}
\setlength{\parindent}{4em}
\setlength{\parskip}{1em}
\renewcommand{\baselinestretch}{1.5}

% These lines describe the environments we will use: Lemma and Theorem
% Although may sound repetitive to declare that our lemma will be displayed as Lemma
% it is needed so that LaTeX can write documents in other languages 
\newtheorem{lemma}{Lemma}
\newtheorem{theorem}{Theorem}

% The commands below change the bold text where it says "Section" into "Question"
\usepackage{titlesec}
\titleformat{\section}
{\normalfont\Large\bfseries}{\thesection}{1em}{} % the section{text}
% will automatically say "Question"

%Commands below change page margins (this much space at the titlepage, etc)
\newlength{\toppush}
\setlength{\toppush}{2\headheight}
\addtolength{\toppush}{\headsep}

%Name and subject of the class
\def\subjnum{EE 98}
\def\subjname{Senior Design}

%Name of the student, university name and which semester
\def\doheading#1#2#3{\vfill\eject\vspace*{-\toppush}%
  \vbox{\hbox to\textwidth{{\bf} \subjnum: \subjname \hfil Celestial Blue}%
    \hbox to\textwidth{{\bf} Tufts University, Spring 2023 \hfil#3\strut}%
    \hrule}}

%Command for the title of the document (Homework 0)
\newcommand{\htitle}[1]{\vspace*{1.25ex plus 1ex minus 0ex}%
\begin{center}
{\large\bf #1}
\end{center}} 

%%%%%%%%%%%%%%%%%%%%%%%%%%%%%%%%%%%%%%%%%%%%%%%%%%%%%%%%%%%%%%%%%%%
% BEGIN DOCUMENT
%%%%%%%%%%%%%%%%%%%%%%%%%%%%%%%%%%%%%%%%%%%%%%%%%%%%%%%%%%%%%%%%%%%
\begin{document}
\newcommand{\keyword}{\textit{neg}}
\newcommand{\Z}{\mathbb{Z}}
\newcommand{\el}{$\ell$}
\doheading{2}{title}{}
\begin{flushleft}
\section{State Transition Matrix}

\begin{equation}
\hat{X}_{k} =
\begin{bmatrix}
r_x  \\
r_y \\
r_z  \\
\dot{r}_x \\
\dot{r}_y \\
\dot{r}_z \\
q_{e2b, i} \\
q_{e2b, j} \\
q_{e2b, k} \\
||q_{e2b}||
\end{bmatrix}
\end{equation}
%100
$$ \hat{\textbf{x}}_{k|k-1} = f(\hat{x}_{k|k-1}, u_k) $$
$f$ is one step of the IMU strapdown. We're not currently able to
solve the Jacobian of $f$.\\
$$P_{k|k-1} = F_{k} P_{k|k-1} F_{k}^T + Q_{k} $$ \\
\text{ $Q$ is a constant diag mat we tune and is the measurement noise.}
\begin{equation} 
x_{k+1} = x_{k} + v dt
\end{equation}
\begin{equation} 
\tilde{\boldsymbol{y}}_{k} = \boldsymbol{z}_{k} - h(\hat{\boldsymbol{x}}_{k|k-1})
\text{ where $ \boldsymbol{z}_{k} $ is elements 0 - 2 of the state vector}
\end{equation}
Measurement matrix $H_{gps}$ is the Jacobian of lla wrt ecef (script provided by tyler). \\
near-optimal Kalman Gain 
\begin{equation}
\boldsymbol{K}_{k} = \boldsymbol{P}_{k|k-1}{{\boldsymbol{H}_{k}^T}}\boldsymbol{S}_{k}^{-1}
\end{equation}
Updated state estimate
\begin{equation}
\boldsymbol{x}_{k|k} = \hat{\boldsymbol{x}}_{k|k-1} + \boldsymbol{K}_{k}\tilde{\boldsymbol{y}}_{k}
\end{equation}
Updated covariance estimate 
\begin{equation}
\boldsymbol{P}_{k|k} = (\boldsymbol{I} - \boldsymbol{K}_{k} {{\boldsymbol{H}_{k}}})\boldsymbol{P}_{k|k-1}
\end{equation}
\end{flushleft}
\end{document}
